\documentclass[12pt, twoside]{book}
% \documentclass[12pt, oneside]{book}  % jednostranna tlac
\usepackage[a4paper,top=2.5cm,bottom=2.5cm,left=3.5cm,right=2cm]{geometry}
\usepackage[utf8]{inputenc}
\usepackage[T1]{fontenc}
\usepackage{graphicx}
\usepackage{url}
\usepackage[hidelinks,breaklinks]{hyperref}
\usepackage[outputdir=build]{minted}
\usepackage{subfigure}
\usepackage{graphicx}
\usepackage[justification=centering]{caption}
\usemintedstyle{friendly}
\graphicspath{ {./images/} }
\linespread{1.25} % hodnota 1.25 by mala zodpovedat 1.5 riadkovaniu
\setcounter{tocdepth}{3}

% -------------------
% --- Definicia zakladnych pojmov
% --- Vyplnte podla vasho zadania
% -------------------
\def\mfrok{2021}
\def\mfnazov{Application level fuzzing}
\def\mftyp{Bachelor Thesis}
\def\mfautor{Matúš Ferech}
\def\mfskolitel{Ing. Peter Gasper}


\def\mfmiesto{Bratislava, \mfrok}
\def\mfodbor{Computer Science}
\def\program{Computer Science }

% Ak je školiteľ z FMFI, uvádzate katedru školiteľa, zrejme by mala byť aj na zadaní z AIS2
% Ak máte externého školiteľa, uvádzajte Katedru informatiky
\def\mfpracovisko{Department of Computer Science}

\begin{document}
\frontmatter


% -------------------
% --- Obalka ------
% -------------------
\thispagestyle{empty}

\begin{center}
  \sc\large
  Comenius University in Bratislava\\
  Faculty of Mathematics, Physics and Informatics

\vfill

{\LARGE\mfnazov}\\
\mftyp
\end{center}

\vfill

{\sc\large
\noindent \mfrok\\
\mfautor
}

\cleardoublepage
% --- koniec obalky ----

% -------------------
% --- Titulný list
% -------------------

\thispagestyle{empty}
\noindent

\begin{center}
\sc
\large
  Comenius University in Bratislava\\
  Faculty of Mathematics, Physics and Informatics

\vfill

{\LARGE\mfnazov}\\
\mftyp
\end{center}

\vfill

\noindent
\begin{tabular}{ll}
Study Programme: & \program \\
Field of Study: & \mfodbor \\
Department: & \mfpracovisko \\
Supervisor: & \mfskolitel \\
\end{tabular}

\vfill


\noindent \mfmiesto\\
\mfautor

\cleardoublepage
% --- Koniec titulnej strany


% -------------------
% --- Zadanie z AIS
% -------------------
% v tlačenej verzii s podpismi zainteresovaných osôb.
% v elektronickej verzii sa zverejňuje zadanie bez podpisov
% v pracach v anglictine anglicke aj slovenske zadanie

\newpage
\thispagestyle{empty}
\hspace{-2cm}\includegraphics[width=1.1\textwidth]{images/assignment}

\hspace{-2cm}\includegraphics[width=1.1\textwidth]{images/assignment-sk}

% --- Koniec zadania

\frontmatter

% -------------------
%   Poďakovanie - nepovinné
% -------------------
\setcounter{page}{7}
\newpage
~

\vfill
{\bf Acknowledgments:} I would like to thank my supervisor Peter Gasper for his mentoring, consultations, insightful knowledge and encouragement in times, when my motivation was plummeting.

I want to thank me for trying to do more rights than wrong, I want to thank me for just being me at all times.

% --- Koniec poďakovania

% -------------------
%   Abstrakt - Slovensky
% -------------------
\newpage
\section*{Abstrakt}

V tejto práci sme sa venovali fuzzovaniu webových aplikacií. Zamerali sme sa najmä na techniku automatizovaného black-box fuzzovania. Na automatický prieskum štruktúry webových aplikacií sme využili OpenAPI špecifikáciu. Na základe tejto špecifikácie sme navrhli a zostrojili generation-based smart fuzzer, ktorý je posiela HTTP requesty spĺňajúce OpenAPI špecifikáciu. Vďaka týmto vlastnostiam OpenAPI Fuzzer dosahuje veľké pokrytie kódu a podarilo sa nám nájsť chyby v celosvetovo používaných webových aplikaciách ako napríklad Kubernetes, Gitea či Vault.


\paragraph*{Kľúčové slová:} fuzzing, OpenAPI, black-box, fuzzer, generation-based
% --- Koniec Abstrakt - Slovensky


% -------------------
% --- Abstrakt - Anglicky
% -------------------
\newpage
\section*{Abstract}
In this work, we researched techniques used for fuzzing web services. We paid attention mainly to the black-box automated fuzzing. For the automatic exploration of the structure of web services, we utilized OpenAPI specification. After knowing the structure of the web service were able to create a generation-based \textit{smart} fuzzer that will construct requests that comply with the OpenAPI specification of the API. Thanks to which we were able to achieve large code coverage and find bugs in such battle-tested production-grade software as Kubernetes, Gitea, or Vault.

\paragraph*{Keywords:} fuzzing, OpenAPI, black-box, fuzzer

% --- Koniec Abstrakt - Anglicky


% -------------------
% --- Obsah
% -------------------

\newpage

\tableofcontents

% ---  Koniec Obsahu

% -------------------
% --- Zoznamy tabuliek, obrázkov - nepovinne
% -------------------

\newpage

\listoffigures
\listoftables

% ---  Koniec Zoznamov

\mainmatter

\input introduction.tex
\input introduction-to-fuzzing.tex
\input from-os-to-api.tex
\input analysis-of-exiting-work.tex
\input architecture-of-out-fuzzer.tex
\input implementation.tex
\input our-findings.tex
\input conclusion.tex

% -------------------
% --- Bibliografia
% -------------------


\newpage

\backmatter

\thispagestyle{empty}
\clearpage

\bibliographystyle{plain}
\bibliography{bibliography}

\input appendix.tex

\end{document}
